\documentclass[letterpaper, oneside]{book}
\usepackage{lingmacros}
\usepackage{tree-dvips}
\usepackage{amsmath}
\usepackage{listings}
\usepackage{graphicx}
\usepackage{xcolor}
\usepackage{mdframed}
\usepackage{fancyhdr}
\usepackage[export]{adjustbox}
\usepackage[skip=10pt]{parskip}
\usepackage{amsfonts}
\pagestyle{plain}



% Configuration

% Set the root directory for images and graphcis.
\graphicspath{{/Users/ruikun/workspace/LatexWorkspace/example/images/}}

%\setlength{\parindent}{0pt}

% Define New Commands

\newcommand*{\vecthree}[3]{
	\begin{bmatrix}
		#1 \\ #2 \\ #3
\end{bmatrix}}

% Define New Environments






\title{My first LaTeX document}
\author{Toby}
\date{September 2023}


\begin{document}
	\maketitle{}
	\tableofcontents


	
	\chapter{Basic}
	
	\section{Paragraph}
	
	This line is supposed to be a very long text. Its purpose is to show how the paragraph works in Latex. As you can see this is a multi-line text.
	
	To start a new paragraph, we can do a blank line in the latex  file. Each new paragraph has a default indent.
	
	This line is to show the effect of the indent mentioned in the previous paragraph.
	
	Some of the \textbf{greatest}
	discoveries in \underline{science} 
	were made by \textbf{\textit{accident}}. hello
	
	Some of the greatest \emph{discoveries} in science 
	were made by accident.
	
	\textit{Some of the greatest \emph{discoveries} 
		in science were made by accident.}
	
	\textbf{Some of the greatest \emph{discoveries} 
		in science were made by accident.}
	
	\section{Listing}
	
	Example: Unordered List \\
	\begin{itemize}
		\item The individual entries are indicated with a black dot, a so-called bullet.
		\item The text in the entries may be of any length.
	\end{itemize}

	Example: Ordered List: \\
	
	\begin{enumerate}
		\item This is the first entry in our list.
		\item The list numbers increase with each entry we add.
	\end{enumerate}
	
	
	
	
	\chapter{Images}
	Example: Use scale parameter \\
	\includegraphics[scale=0.6]{One_Piece_Logo.png}  

	Example: Use max width and linewidth from adjustbox package. \\
	\includegraphics[max width=\linewidth]{One_Piece_Logo.png}  
	
	Example: Use max width and textwidth from adjustbox package. \\
	\includegraphics[max width=\textwidth]{One_Piece_Logo.png}  
	
	Example: Use figure and reference.
	\begin{figure}[h]
		\centering
		\includegraphics[width=0.75\textwidth]{One_Piece_Logo.png}
		\caption{A nice plot.}
		\label{fig:mesh1}
	\end{figure}
	
	As you can see in figure \ref{fig:mesh1}, the function grows near the origin. This example is on page \pageref{fig:mesh1}.
	
	


	\chapter{Math}


Example: Inline Math formula:  \\

In physics, the mass-energy equivalence is stated 
by the equation $E=mc^2$, discovered in 1905 by Albert Einstein. \\


Example: Inline Math formula 2: \\

\begin{math}
	E=mc^2
\end{math} is typeset in a paragraph using inline math mode---as is $E=mc^2$, and so too is \(E=mc^2\).

Example: Math Block: \\

The mass-energy equivalence is described by the famous equation
\[ E=mc^2 \] discovered in 1905 by Albert Einstein. 

In natural units ($c = 1$), the formula expresses the identity
\begin{equation}
	E=m
\end{equation}
	
	\include{chapter_table}
	\include{chapter_command_and_environment}
	
	
	
\end{document}

}
